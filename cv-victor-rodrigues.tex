\documentclass[12pt,a4paper]{article}

\usepackage{cvstyle}


\begin{document}

\noindent\makebox[\linewidth][c]{%
  \noindent\begin{minipage}{.3\linewidth}
    \bfseries\Large
    Victor Rodrigues
  \end{minipage}%
  \begin{minipage}{.5\linewidth}
    \begin{flushright}
    %Sheffield, UK\\ \vspace{0.2cm}
    {\ttfamily 07599247587}\\
    \href{mailto:victorlampreia@gmail.com}{\ttfamily victorlampreia@gmail.com}\\
    \url{github.com/vlampreia}
    \end{flushright}
  \end{minipage}
}


%-------------------------------------------------------------------------------
\nheader{Education}

\nentry{2013}{2017}
       {University of Manchester}
       {B.Sc. Computer Science}
{%
  \begin{itemize}
    \item Achieved {\bfseries First-class} in the second year.

    \item Built an automatic birdsong recognition system based on spectrogram
      image recognition using computer vision techniques and random forests on
      cross-correlation mapping results using
      {\bfseries Python}.
      Achieved an accuracy of 89\% on a limited dataset.

    \item Achieved a First-class for first year team project, a web application
      for creating and sharing music playlists -- exploring metadata visualisation.

    \item Developed core team-working, management and communication skills
      through enquiry-based learning. Exposed to diverse technologies.

  \end{itemize}
}

%-------------------------------------------------------------------------------
\nheader{Experience}

\nentry{Sep 2015}{Aug 2016}
       {CERN, Switzerland}
       {Intern Technical Student Software Engineer}
{%
  Supervised self-driven development of modules for the Siemens {\bfseries WinCC-OA}
  SCADA software which is used across CERN's infrastructure, from
  electrical networks to cryogenics.
  Software was written in a proprietary interpreted {\bfseries C}-like language
  within platform constraints with a {\bfseries TDD} approach.

  \begin{itemize}
    \item Built a feature-complete distributed RPC framework for WinCC-OA
      from the ground-up.

    \item Developed a specialised database-oriented communication system based
      on lock-free ring-buffers as a foundation for the client-broker architecture.

    \item Designed a robust RPC protocol for concurrent multi-system and multi-executor
      support -- semi-resilient to client and server failure.

    \item Developed integrated diagnostic tooling to gain insights into RPC transactions in
      real-time.

    \item Developed a WinCC-OA integrated module for modbus network administration,
      through direct application of the developed RPC framework.
  \end{itemize}
}

\nentry{Jul 2014}{Aug 2015}
       {RefME, London}
       {Software Developer}
{%
  Developed backend {\bfseries Node.js} modules for a multi-platform
  bibliographic referencing app within an {\bfseries Agile} and
  {\bfseries TDD}-focused environment as part of a successful startup.


  \begin{itemize}
    \item Developed extensible modules for porting bibliographic data between
      internal JSON and third-party formats such as
      \emph{BibTeX},
      \emph{RIS},
      \emph{EverNote} XML, and
      \emph{DOCX}.

    \item Wrote web scrapers for detecting and retrieving bibliographic data
      from public websites such as
      \emph{JSTOR},
      \emph{BBC News}, and
      \emph{NCBI}.

    \item Ported API lookup modules from {\bfseries Ruby on Rails} to Node.js.

  \end{itemize}
}

\nentry{Jun 2011}{Aug 2011}
       {Museu da Lourinh\~a, Portugal}
       {Intern Database and Systems Administrator}
{%
  Assisted in modernisation efforts to improve the museum's previously manual
  record keeping and management procedures, including database design, system
  administration and network design.
}

\pagebreak
%-------------------------------------------------------------------------------
\nheader{Technical Knowledge and Experience}%
%
\hfill (in decreasing order of proficiency)\\
\vspace{0.7cm}

\begin{tabular}{>{\bfseries}l l}
  Languages   & NodeJS/Javscript, C, Java, Python, C++, HTML/CSS, ARM ASM\\
  Databases   & PostgreSQL, MySQL, Neo4j (SPARQL)\\
  SVC         & Git, SVN, Hg\\
  Tooling     & GNU, Clang, Vim, \\
  Platforms   & GNU/Linux, MS Windows, WinCC-OA, OpenBSD\\
  Methodologies  & TDD, Agile (Scrum), Standard Waterfall
\end{tabular}

%-------------------------------------------------------------------------------
\nheader{Hackathons}

\nentry{Nov 2016}{}
        {GreatUniHack}
        {MLH, Manchester}
{%
  \begin{itemize}
    \item {\bfseries Winner} of first category
  \end{itemize}

  Developed a convolutional neural network model which generates 2d levels for
  an open-source Super Mario clone. The model is trained on existing levels by
  automatic conversion to an ASCII representation.
}

\nentry{Feb 2015}{}
       {StacsHack}
       {University of St. Andrews, Scotland}
{%

  Developed a computer-vision application which uses machine-learning techniques
  to recognise and attribute a score to a correct series of physical movements
  by the player.

}

\nentry{Oct 2014}{}
       {24 Hour Coding Competition}
       {HackManchester, Manchester}
{%
  Developed a web API in Node.js enabling users to compete and challenge
  each-other for charity with a set wager, with a team of four within 24 hours.
  The API would be integrated into online games through a REST API\@.
}

\nentry{Oct 2014}{}
       {PayPal BattleHack}
       {PayPal, London}
{%
  Developed a multi-stage authentication framework in Node.js with a team of
  four within 24 hours. It would provide an alternative method of authentication
  by use of trusted QR codes and tokens.
}

\nentry{Jun 2014}{}
       {HACKcelerator Programme}
       {AngelHack, London}
{%
  \begin{itemize}
    \item{\bfseries Winner} for entry into a pre-accelerator
      programme for startups
  \end{itemize}

  Developed a multi-user cooperative holiday planning iOS app, with a Node.js
  backend with a team of four within 24 hours. The app retrieves personal
  information from Facebook, a set of user constraints for budgeting, and
  automatically generates an itinerary with suggested hotels based on their
  proximity to events with potential interest to the user, as well as
  transportation options.
}

\nentry{Mar 2014}{}
       {Greater Manchester Data Synchronisation Programme}
       {OpenData Manchester}
{%
  \begin{itemize}
    \item {\bfseries Winner of four categories}
  \end{itemize}

  Developed two web applications for graphical data visualisation with a team of
  four within 24 hours, integrating multiple open datasets and external APIs as
  part of a pioneering initiative towards free-flowing open-data amongst the
  the public sector organisations within Greater Manchester.
}

\end{document}
