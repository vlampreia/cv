\documentclass[12pt,a4paper]{article}

\usepackage{cvstyle}


\begin{document}

\noindent\makebox[\linewidth][c]{%
  \noindent\begin{minipage}{.3\linewidth}
    \bfseries\Large
    Victor Rodrigues
  \end{minipage}%
  \begin{minipage}{.5\linewidth}
    \begin{flushright}
    %Sheffield, UK\\ \vspace{0.2cm}
    {\ttfamily 07599247587}\\
    \href{mailto:victorlampreia@gmail.com}{\ttfamily victorlampreia@gmail.com}\\
    \url{github.com/vlampreia}
    \end{flushright}
  \end{minipage}
}


%-------------------------------------------------------------------------------
\nheader{Education}

\nentry{2013}{2017}
       {University of Manchester}
       {B.Sc. Computer Science}
{%
  \begin{itemize}
    \item Achieved {\bfseries First-class honours} in the second year;

    \item Currently exploring automatic birdsong recognition using
      computer vision and machine-learning techniques.

    \item Achieved a First-class for first year team project, a web application
      written using {\bfseries PHP} and {\bfseries JavaScript} for creating and
      sharing music playlists, exploring new methods for music data
      visualization.

    \item Developed core team-working, management and communication skills
      through enquiry-based learning. Exposed to diverse technologies.

  \end{itemize}
}


%-------------------------------------------------------------------------------
\nheader{Experience}

\nentry{Sep 2015}{Aug 2016}
       {CERN, Switzerland}
       {Intern Technical Student Software Engineer}
{%
  Developed exetensions for the specialised Siemens/ETM {\bfseries WinCC-OA}
  SCADA software used across CERN's infrastructure, used in applications ranging
  from electrical networks to cryogenics to magnet protection systems. Software
  was developed using a proprietary interpreted {\bfseries C}-like language.

  \begin{itemize}

    \item Developed a feature-complete distributed RPC framework for WinCC-OA
      from the ground-up;

    \item Follows broker-client architecture, allows balancing of multiple
      concurrently executing worker processes across multiple systems;

    \item Developed underlining ringbuffer broker-client communication system
      for the framework;

    \item Developed existing use-case for the framework, a WinCC-OA
      configuration application for SCADA technicians.

  \end{itemize}
}

\nentry{Jul 2014}{Aug 2015}
       {RefME, London}
       {Software Developer}
{%
  Developed backend {\bfseries Node.js} modules for a bibliographic referencing
  application within a strong {\bfseries Agile} and {\bfseries TDD} environment.

  \begin{itemize}

    \item Developed core modules for porting bibliographic data between
      internal JSON and third-party formats including
      \emph{BibTeX},
      \emph{RIS},
      \emph{EndNote} XML,
      \emph{Evernote} XML, and
      \emph{.docx};

    \item Wrote web scrapers for detecting and retrieving bibliographic data
      from public websites such as
      \emph{JSTOR},
      \emph{BBC News},
      \emph{NCBI}, etc.;

    \item Ported API lookup components from an original Ruby codebase to Node.js
      modules.

  \end{itemize}
}

\nentry{Jun 2011}{Aug 2011}
       {Museu da Lourinh\~a, Portugal}
       {Intern Database and Systems Administrator}
{%
  Assisted in modernization efforts to improve the museum's previously manual
  record keeping and management procedures, including database design, system
  administration and network design.
}

\pagebreak
%-------------------------------------------------------------------------------
\nheader{Technical Knowledge and Experience}

In order of proficiency:\\

\begin{tabular}{>{\bfseries}l l}
  Languages   & NodeJS/Javscript, C, Java, C++, Python, HTML/CSS, ARM ASM\\
  Databases   & PostgreSQL, MySQL, Neo4j (SPARQL)\\
  SVC         & Git, SVN\\
  Tooling     & Vim, GDB, Eclipse\\
  Platforms   & GNU/Linux, MS Windows, WinCC-OA, OpenBSD\\
  Methodology & TDD, Agile (Scrum)
\end{tabular}

%-------------------------------------------------------------------------------
\nheader{Hackathons}

\nentry{Feb 2015}{}
       {StacsHack}
       {University of St. Andrews, Scotland}
{%

  Developed a computer-vision application which uses machine-learning techniques
  to recognize and attribute a score to a correct series of physical movements
  by the player.

}

\nentry{Oct 2014}{}
       {24 Hour Coding Competition}
       {HackManchester, Manchester}
{%
  Developed a web API in Node.js enabling users to compete and challenge
  eachother for charity with a set waiger, with a team of four within 24 hours.
  The API would be integrated into online games through a REST API\@.
}

\nentry{Oct 2014}{}
       {PayPal BattleHack}
       {PayPal, London}
{%
  Developed a multi-stage authentication framework in Node.js with a team of
  four within 24 hours. It would provide an alternative method of authentication
  by use of trusted QR codes and tokens.
}

\nentry{Jun 2014}{}
       {HACKcelerator Programme}
       {AngelHack, London}
{%
  \begin{itemize}
    \item{\bfseries Winner} for entry into a pre-accelerator
      programme for startups
  \end{itemize}

  Developed a multi-user cooperative holiday planning iOS app, with a Node.js
  backend with a team of four within 24 hours. The app retrieves personal
  information from Facebook, a set of user constraints for budgeting, and
  automatically generates an iteniary with suggested hotels based on their
  proximity to events with potential interest to the user, as well as
  transportation options.
}

\nentry{Mar 2014}{}
       {Greater Manchester Data Synchronization Programme}
       {OpenData Manchester}
{%
  \begin{itemize}
    \item {\bfseries Winner of four categories}
  \end{itemize}

  Developed two web applications for graphical data visualization with a team of
  four within 24 hours, itegrating multiple open datasets and external APIs as
  part of a pioneering initiative towards free-flowing open-data amongst the
  the public sector organizations within Greater Manchester.
}

\end{document}
